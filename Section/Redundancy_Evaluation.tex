\subsection{Evaluation}\label{redundancy_evaluation}
In this section we will present the results of applying our approach to analyse USs in 19 by CRF tool annotated backlog datasets. 

The purpose of this section is to assess the effectiveness of our approach in analysing redundancy between USs. This section serves several key objective:
\begin{itemize}
	\item Validation of the approach: Demonstrating the validity and reliability of our proposed methodology for identifying and categorising redundancy in USs.
	\item Evaluation of the results: Presenting the results of our analysis across multiple datasets, highlighting the prevalence of full and partial redundancy for both the main and benefit parts in USs.
	\item Gaining insights: Gaining insights into the nature and extent of redundancy in USs and its potential impact on software development processes.
	\item Comparison and analysis: comparing results from different data sets, identifying patterns or trends and analysing variations in redundancy levels.
\end{itemize}
\subsubsection*{Dataset of Backlogs}
For a comprehensive assessment, we applied our approach to 19 backlog datasets presented by Mosser et al. \footnote{https://github.com/ace-design/nlp-stories/tree/3b025710225347355eed778974939f7cb3c6d4e4/nlp/nlp\_outputs}. Mosser et al. applied the CRF approach to these publicly available requirements datasets\ref{requirementsdatasets}.

It is also worth noting that some backlog datasets (g02, g13, g17, g27) did not follow the expected sentence structure, which is why we did not include them in the evaluation results.
Table \ref{tb:backlogs} shows the project number of each data set and the count of USs.
 %	- You have to describe all that in your work, how do you define Main Partial/Full and Benefit Partial/Full, so how do you decide that. And generating this JSON_Report and filling this table is all interesting implementation information.
 
 %When evaluating, I use tools on a large scale and I go through all the cases and look at what comes out of the use case as a meaningful result. As an evaluation, I will enter all the datasets we have as input and then what comes out will be 
\begingroup
\centering
\scriptsize
\renewcommand{\arraystretch}{1,5} 
%\keepXColumns
\begin{tabularx}{\linewidth}{l|XXXXXXXXXXXXXXXXXXX|X}
	Item&	1&	2&	3&	4&	5&	6&	7&	8&	9&	10&	11&	12&	13&	14&	15&	16&	17&	18&	19&	\\
	\hline
	%Project Name&loudoun&recycling&openspending&frictionless&scrumalliance	&nsf&camperplus&datahub&mis&neurohub&alfred&badcamp&rdadmp&archivesspace	&unibath&duraspace&racdam&culrepo&zooniverse&\\
	Project Nr.&	g03	&g04	&g05	&g08	&g10	&g11	&g12	&g14	&g16	&g18	&g19	&g21	&g22	&g23	&g24	&g25	&g26	&g27	&g28	&Total USs\\
	\hline
	Total USs&	57&	51	&53	&66	&97	&73	&54	&67	&66	&102	&137	&69	&83	&56	&53	&100	&100	&114	&60	&1458 \\
	\caption{Project number and count of USs contained in each backlog dataset}\label{tb:backlogs}
\end{tabularx}	
	\endgroup
\subsubsection*{Methodology}
In this section, we recap the methodology employed to analyse redundancy between USs. We utilized a systematic approach that involved several key steps:
ll\begin{itemize}
	\item Data Collection: We collected a total 19 datasets containing USs from various software development projects.
	\item Preprocessing: Using the generated JSON report file for each dataset, we create a VBA script called \textit{extractFromJSONFiles}\footnote{https://github.com/amirrabieyannejad/USs\_Annotation/tree/main/Skript/extractFromJSONFiles} to iterate through all the JSON report files and extract the information such as \enquote{US-pair}, \enquote{US-text}, \enquote{total redundancy clauses}, \enquote{main redundancy clauses}, \enquote{benefit redundancy clauses} and \enquote{project number} and collect them all in an Excel file to perform further analyses.
	\item Identification of main and benefit parts: Each US was divided into its main part, which is the core functionality, and its benefit part, which describes the value to the persona.
	\item redundancy analysis: A redundancy analysis was performed between the main and the benefit part of the US-pairs. The redundancy was categorised as either \enquote{Full} or \enquote{Partial} redundancy.
	
	A US-pair was categorised as \enquote{Full redundancy} if all clauses were identical in either the \enquote{Main part} or the \enquote{Benefit part}. This indicates Full Redundancy between USs in a particular part.
	
	\enquote{Partial Redundancy} was found when only some clauses were redundant in either the Main or Benefit parts, while others were unique. This indicates overlapping clauses, but not full redundancy.
	
	\begin{example}
		For example, the following US-pair is identified as \enquote{Full Redundancy} in the main part:
		
		\textit{user\_story\_01:} \enquote{\#g14\# as a \#publisher\#, i want to \#publish\# a \#dataset\#, so that i can view just the dataset with a few people.}
		
		\textit{user\_story\_02:} \enquote{\#g14\# as a \#publisher\#, i want to \#publish\# a \#dataset\#, so that i can share the dataset publicly with everyone.}	
		 As all clauses in the main part are identical between USs, therefore, we call it \enquote{Full Redundancy} in \enquote{Main} part.
	\end{example}
	\begin{example}
		Following US-pair is identified as \enquote{Partial Redundancy} in main part, as only some clauses are redundant:
		
		\textit{user\_story\_09:} \#g04\# as a \#user\#, I want to be able to \#view\# a \#map display\# of the public recycling bins around my \#area\#.
		
		\textit{user\_story\_10:} \#g04\# as a \#user\#, I want to be able to \#view\# a \#map display\# of the special waste drop off sites around my \#area\#.
		
		As we can see, there are some redundancy clauses between the USs, but not all, \textit{e.g.} Noun\enquote{public recycling bins} vs. \enquote{special waste drop-off centres}, which leads us to evaluate this as \enquote{partial redundancy} in \enquote{Main} part.
	\end{example}
	\begin{example}
		Following US-pair is identified as \enquote{Full Redundancy} in \enquote{Benefit} part:
		
		\textit{user\_story\_02:} \#g05\# as a data publishing user, I want to be able to edit the model of data I have already imported, so that I can \#fix\# \#bugs\# or \#make\# \#enhancements\# in the \#API\# built for my \#data\#.
		
		\textit{user\_story\_07:} \#g05\# as a data publishing user, I want to be able to edit the data source of data I have already imported, so that i can \#fix\# \#bugs\# or \#make\# \#enhancements\# in the \#API\# built for my \#data\#.
		
		As we can see, all clauses in the benefit part are identical between USs, therefore, we call it \enquote{Full Redundancy} in \enquote{Benefit} part.
	\end{example}
	\begin{example}
	Following US-pair is identified as \enquote{Partial Redundancy} in benefit part, as only some clauses are redundant:
	
	\textit{user\_story\_17:} \#g03\# as a staff member, I want to manage approved proffers, so that I can \#ensure\# \#compliance\# with and satisfaction of the proffer in the future.
	
	\textit{user\_story\_30:} \#g03\# as a staff member, I want to manage affidavits, so that I can \#ensure\# \#compliance\# with the requirements prior to the hearing.
	
	As we can see, there are some redundancy clauses between the USs, but not all, \textit{e.g.} Clause\enquote{proffer}and\enquote{satisfaction} vs. \enquote{requirements}and\enquote{hearing}, which leads us to evaluate this as \enquote{Partial Redundancy} in \enquote{Benefit} part.
	\end{example}
	
	%\item Preprocessing: \item preprocessing: We transform each data set into graph transformation rules, apply CDA and generate a textual report in which each redundancy clause between US-pairs is marked with a hash symbol. Additionally,  the number of founded redundancies in the main and benefit sections is also determined.
\end{itemize}
%\subsection{Evaluation}\label{redundancy_evaluation}
 %	- You have to describe all that in your work, how do you define Main Partial/Full and Benefit Partial/Full, so how do you decide that. And generating this JSON_Report and filling this table is all interesting implementation information.
 
 %When evaluating, I use tools on a large scale and I go through all the cases and look at what comes out of the use case as a meaningful result. As an evaluation, I will enter all the datasets we have as input and then what comes out will be 


\begin{figure}[h]
\center
\includegraphics{Providing_early_feedback_at_the_backlog_level}
\caption{Providing early feedback at the backlog level \cite{mosser2022modelling}}\label{fig:early_feedback}
\end{figure}


 
%\subsection{Evaluation}\label{redundancy_evaluation}
 %	- You have to describe all that in your work, how do you define Main Partial/Full and Benefit Partial/Full, so how do you decide that. And generating this JSON_Report and filling this table is all interesting implementation information.
 
 %When evaluating, I use tools on a large scale and I go through all the cases and look at what comes out of the use case as a meaningful result. As an evaluation, I will enter all the datasets we have as input and then what comes out will be 


\begin{figure}[h]
\center
\includegraphics{Providing_early_feedback_at_the_backlog_level}
\caption{Providing early feedback at the backlog level \cite{mosser2022modelling}}\label{fig:early_feedback}
\end{figure}


 
%\subsection{Evaluation}\label{redundancy_evaluation}
 %	- You have to describe all that in your work, how do you define Main Partial/Full and Benefit Partial/Full, so how do you decide that. And generating this JSON_Report and filling this table is all interesting implementation information.
 
 %When evaluating, I use tools on a large scale and I go through all the cases and look at what comes out of the use case as a meaningful result. As an evaluation, I will enter all the datasets we have as input and then what comes out will be 


\begin{figure}[h]
\center
\includegraphics{Providing_early_feedback_at_the_backlog_level}
\caption{Providing early feedback at the backlog level \cite{mosser2022modelling}}\label{fig:early_feedback}
\end{figure}


 
%\input{Section/Redundancy_Evaluation}



