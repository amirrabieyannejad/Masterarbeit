%benefit of graph transformation
\newpage
\section*{ Abstract}
\emph{User stories} (USs), the basic building blocks of software development, serve as precise and testable descriptions of a software's functionality. Within the dynamic framework of \emph{agile development}, these USs become a frequently used requirements notation in agile projects\cite{wang2014role}, which is usually written informally in plain text and managed in the \textit{product backlog}, which serves as a repository for prioritising and tracking development tasks.

As the amount of USs increases, conflicts or redundancies between them are inevitable. If a user story (US) requires the deletion of a component that is essential for the successful execution of another US, we are dealing with a conflict, or if a US (or some elements or parts of it) is a syntactic duplication of another US, we are dealing with redundancy.

In addition, changing an existing requirement or adding a new requirement to the existing product backlog in agile software development can also cause conflicts or redundancies due to changes in the needs and concerns of \emph{system stakeholders}. This can result in a wide range of inconsistencies, as requirements are raised by multiple stakeholders involved in product development to achieve different functions.

Effectively recognising these conflicts and redundancies is fundamental for development teams. By addressing these issues, teams can provide additional value to users, adapt to changing requirements and maintain consistency between USs. 

Normally, agile methods such as \emph{Scrum} encourage cross-functional collaboration and daily stand-up meetings as mechanisms to address and mitigate redundancies and conflicts in a timely manner. However, this approach can be time and resource consuming. In cases where the backlog is very extensive, recognising existing redundancies and conflicts between USs can become a complex undertaking.

Since there is no method for automatically identifying redundancies and conflicts between requirements written in natural language in agile software development, in this thesis, we want to present two approaches for analysing redundancies and conflicts between USs.

The first approach analyses redundancies by taking annotated USs as input (instead of US text) and using \textit{graph transformation} (GT), in particular \textit{Henshin} and its \textit{Conflict and Dependency Analysis} (CDA) tool, to detect potential redundancies in a syntactic way.

The second approach deals with analysing conflicts between USs, also using annotated USs as input. By utilizing \textit{Natural Language Processing} (NLP) technique, particularly VerbNet, and a specially implemented tool, we recognize potential conflicts in a semantic way.

We apply both approaches to 19 annotated backlog datasets and create comprehensive reports for each dataset. Upon evaluation, we find that the results of both approaches are satisfactory. However, we also notice that the quality of the USs and their annotations significantly influences the effectiveness of the results.
