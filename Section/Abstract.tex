%benefit of graph transformation
\newpage
\section*{ Abstract}
%Natural Language user stories is user-oriented and used in everyday life, and commonly used in the early phase of software development. Despite many problems such as the lack of formality, structure and ambiguity of natural languages, they are still regarded as one of the most important communication medium between developers and customers. Thus we propose a systematic process and a supporting tool for detecting and managing requirements conflicts in natural language.
%During the development process, a change to a model may result in an inconsistent model that must at some point be repaired. Detecting conflicts between requirement is crucial for successful software development. In addition, there is no overall process that covers all the steps for managing conflict.
\emph{User stories}, the basic building blocks of software development, serve as precise and testable descriptions of a software's functionality. Within the dynamic framework of \emph{agile development}, these user stories are usually written informally in simple text and managed in the product backlog, which serves as a repository for prioritising and tracking development tasks.

\emph{Behaviour Driven Development} (BDD), a special approach in the field of agile software development, places a strong emphasis on the iterative implementation of user stories. The sequence of user stories in BDD is a central aspect of the methodology. The correct sequence not only has an impact on the efficiency of development, but also on the overall success of the project. By effectively prioritising and sequencing user stories, development teams can deliver incremental value to users, respond to changing requirements and ensure that the most critical features are addressed first.

User stories often have dependencies on each other, leading to potential conflicts if one user story requires the deletion of a component that is essential for the successful execution of another user story, or if one user story introduces an element that runs counter to the realisation of another user story and thus prevents it. In addition, changing an existing requirement or adding a new requirement to the existing \emph{product backlog} in agile software development can also cause conflicts due to changes in the needs and concerns of \emph{system stakeholders}. This can result in a wide range of inconsistencies due to conflicting requirements, as requirements are raised by multiple stakeholders involved in product development to achieve different functions.

To minimise the occurrence of conflicts, teams should systematically identify and document dependencies between user stories within their backlog. Agile methods such as \emph{Scrum} encourage cross-functional collaboration and daily stand-up meetings as mechanisms to address and mitigate dependencies and conflicts in a timely manner. However, this approach can be time and resource consuming. In cases where the backlog is very extensive, recognising existing conflicts between user stories can become a complex undertaking.

\emph{Model-based software engineering} is a suitable method for coping with the ever-increasing complexity of software development processes. \emph{Graphs} and \emph{graph transformations} have proven to be useful for visualising such models and their changes.

Since there is no process in agile software development to systematically identify and manage conflicts between requirements created in natural language, this thesis aim to presents a well-structured workflow that uses a collection of techniques and tools from different domains to accelerate the automatic detection of conflicts and dependencies in natural language user stories.