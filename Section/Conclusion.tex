\section{Conclusion}\label{conclusion}
In this thesis, we addressed the challenges of managing redundancies and conflicts in USs within agile software development. Our work focused on developing two comprehensive workflows that aim to increase the efficiency of US management through the use of advanced tools and methods specially graph-based annotation of USs.

First, we proposed a redundancies analysis framework that combines the Doccano tool, Henshin, and the CDA tool. By analysing 19 different backlog datasets, we have shown that our approach can effectively detect and report redundancies. Our results emphasise the importance of well-formulated and accurately annotated USs in detecting redundancies. In particular, we found that the benefit parts of USs tend to be more redundant than the main parts, indicating a commonality of targets across different USs.

The redundancies analysis helps to recognise functionally identical USs and to group related USs in order to reduce the project backlog and improve project management.

Finally, we developed a framework for conflicts analysis that utilises VerbNet as a computational lexical resource. This method enabled us to systematically identify and report conflicts by categorising the verbs of the main parts of USs into four action-annotations such as "delete", "create", "forbid" and "preserve". Our evaluation of 19 backlog datasets revealed that high-quality USs that do not contain ambiguous verbs and nouns are crucial for effective conflicts analysis.

Recognising and resolving conflicts between USs ensures that project requirements are consistent and coherent, preventing potential problems during implementation.

The study confirms that both syntactic and semantic analyses are essential for the management of USs in agile development. While the detection of redundancies benefits from syntactic analysis, the detection of conflicts relies heavily on semantic understanding. The quality and clarity of the annotated USs are central to the success of these approaches.

Future research can build on our results by further analysing the dependencies between USs in both semantic and syntactic ways. Furthermore, we want to compare our approaches with NLP-based techniques that recognise redundancies between USs (as described in works such as \cite{duszkiewicz2022identifying,jurischsimilarity,levin1993english}). 

The insights gained from this research provide valuable guidance for improving current practices in software project management and for shaping future advances in this area.

In summary, our work contributes to streamlining software development processes by introducing robust frameworks for redundancies and conflicts analysis, thereby supporting the iterative and incremental nature of agile development.