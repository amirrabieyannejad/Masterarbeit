\section{Analysing Redundancy}\label{redundancy}
% some explenation here about what will be explain here an why
%(Analyse von Redundanz als title und dann im Text Schreiben, dass es ein syntaktische Analyse ist
In this section, we present an approach for syntactically analysing redundancy using the CRF tool, the Henshin and the CDA tool.

In section \ref{redundancy_requirement} we present the requirements and functional needs that are used as input for the design phase to fulfil the requirements. In section \ref{desing} we explain the design decisions of the workflow shown in figure \ref{fig:design_phases} and explain how the architecture is structured and what the components and their classes look like. 
\begin{figure}[h]
	\centering 
	\includegraphics[scale=0.35]{sequence_diagram}
	\caption{Design phases}\label{fig:design_phases}
\end{figure} 

In Section \ref{redundancy_implementation} comes the implementation to show what we implemented and in Section \ref{redundancy_test} we show how we tested it. Finally, we apply our approach to 19 datasets of backlogs annotated with the CRF tool \footnote {https://github.com/ace-design/nlp-stories}. In Section \ref{redundancy_evaluation} we evaluate the results.
\subsection{Requirements}\label{redundancy_requirement}
In order to accomplish the analysis of redundancy in USs we try to address following requirements:
\begin{itemize}
\item As a member of a project group, I want to syntactically analyse USs that belong to a specific backlog, so that redundancies between USs can be recognise and manage accordingly.
\item As a member of a project group, I would like to have a collection of US-pairs as a report that syntactically have the same clauses in main and benefit parts of USs, so that I can change them if necessary.
\item As a member of a project group, I want to filter the report and only see the redundancy clauses that contain \enquote{Action} (as a verb) and \enquote{Entity} (as a noun) in US-pair, so that I can reduce the count of potential redundancy pairs in the report.
\item As a member of a project group, I want to mark found redundancy clauses with a hash symbol (\#) and show those that have a redundancy in \enquote{Persona} (as a noun) and \enquote{Action} (as a verb), which is called \enquote{Triggers}, so that I can better see if the persona in is also recognised as a redundancy.
\item As a member of a project group, I want to mark founded redundancy clauses with a hash symbol (\#) and show those that contain two \enquote{Entity}(as a noun), which is called \enquote{Contains}, so that I can better see whether the contained entity is also recognised as a redundancy.
\item As a member of a project group, I would like to have a redundancy report that shows founded US texts in US-pairs and adds a hash symbol (\#) at the beginning and end of the founded words as a marker, so that I can better see which words are redundant in US-pairs.
\item As a member of a project group, I want to see how many redundancy clauses were founded in each US-pair, so that I can aggregate each founded redundancy US-pair for further statistical purposes on that basis.
\item As a member of a project group, I want a table at the top of the redundancy report that lists the US-pairs and the number of redundancy clauses contained in each pair, so that I can quickly see all the US-pairs founded and the number of redundancy clauses. 
\item As a member of a project group, I want to know whether the redundancy clauses belong to the main or benefit part of the US, so that I can use it for further statistical purposes.
\end{itemize}
% here should explain what was the reiquirement and why

\subsection{Design}\label{desing}
% what are the design decisions of this workflow that I had at the beginning, then describe what happens step by step, what translation I made, what is the main design decision you made for this translation, what question and that is part of the design
%In Desing should be how the architecture is based on which components and what is my own component, what does it use, what does the component look like, what class is there, is the class design, so everything comes from the Desing chapter
In this section, the workflow process shown in figure \ref{fig:design_phases} is described step by step. It also explains which design decisions were made for this workflow and how the architecture is based on which components and classes.

To fulfil the requirements mentioned in \ref{redundancy_requirement}, we use the CRF tool \cite{mosser2022modelling} for the annotation of backlogs and the Henshin API \cite{arendt2010henshin} under Java programming language for the automatic generation of rules for each US in the backlog. Moreover, we use Henshin's CDA(conflict and dependency analysis) feature \cite{mens2007analysing} to automatically recognise redundancy US-pairs.\\\\
A detailed, step-by-step description of the workflow is given below:
\subsubsection*{CRF Tool}\label{workflow_crf}
As input, we receive a graph-based model generated by the CRF tool, which represents the refined and annotated dataset for the recognition of \emph{entities}, \emph{actions}, \emph{persons} and \emph{benefits} of USs \cite{mosser2022modelling}.

Mooser et al. have linked each \emph{Persona} to each \emph{Primary Action} as \emph{Trigger} relationships, each \emph{Primary Actions} to each \emph{Primary Entity} as \emph{Target} relationships and each \emph{Primary/Secondary Entity} to each \emph{Primary/Secondary Entity} implying a \emph{Contains} relationship\cite{arulmohan2023extracting}. As output we receive a JSON-file which contains all annotated USs separated by each backlogs.
\subsubsection*{Identifying USs in JSON-File}\label{workflow_nummerize_us}
Annotated USs in each JSON file have no identifier. To distinguish USs, we use a Python script called \textit{nummerise\_us.py} \footnote{https://github.com/amirrabieyannejad/USs\_Annotation/tree/main/Script/numberise\_us}, which receives JSON files as input and adds a JSON object named \enquote{US\_Nr} with an identifier as value (e.g. user\_story\_01) to each US and returns the JSON files as output.
\subsubsection*{Creating Ecore Meta-Model}\label{workflow_ecore}
To be able to create rules in Henshin, an Ecore (meta)-model should be available. Ecore is the core (meta)-model at the heart of the EMF (Eclipse Modelling Framework). It enables the formulation of other models by utilising its constructs.

Accordingly, we create an Ecore meta-model as shown in Figure \ref{fig:ecore_meta_model}, which is inspired by the meta-model shown in Figure \ref{fig:conceptual_metamodel} and corresponds to the JSON-objects in the JSON-file as follows:
\begin{itemize}
	\item \textit{Persona} as a class in the meta-model corresponds to the JSON-object \enquote{Persona} in the JSON-file.
	\item \textit{Entity} as an abstract class, from which \textit{Primary/Secondary Entity} inherits as a class in the meta-model, corresponds to the JSON-object \enquote{Entity}, which contains two JSON-arrays, namely \enquote{Secondary/Primary Entity} in the JSON-file.
	\item \textit{Action} as an abstract class and \textit{Primary/Secondary Action} as an inherited class in the meta-model correspond to the JSON-object \enquote{Action}, which contains two JSON-arrays, namely \enquote{Secondary/Primary Action} in the JSON-file.
	\item \textit{Benefit} as a class in the meta-model, which also has an attribute called \enquote{text} that corresponds to the JSON-object \enquote{Benefit} in the JSON-file.
	\item \textit{Story} as a class in the meta-model that contains text from US, which also has an attribute called \enquote{text} that corresponds to the JSON-object \enquote{Text} in the JSON-file.
	\item Abstract class \textit{NamedElement} has attribute \textit{name}, which Primary/Secondary Action/Entity inherit from it, which corresponds to the value of \textit{Primary/Secondary Action/Entity} in JSON-file.
	\item \textit{Edge} with the name \textit{triggers} between Persona and Primary Action in the meta-model, which corresponds to the JSON-array \enquote{Triggers}, where each JSON-array in it contains a pair, the first element corresponding to the \textit{Persona} and the second to the \textit{Primary Action}.
	\item \textit{Edge} named \textit{targets} between Primary/Secondary Action and Primary/Secondary Entity in the meta-model, which corresponds to the JSON-array \enquote{Targets}, where each JSON-array has a pair, the first element corresponding to \enquote{Primary/Secondary Action} and the second element corresponding to \enquote{Primary/Secondary Entity}.
	\item \textit{Edge} named \textit{contains} between Primary/Secondary Entity and itself in the meta-model, which corresponds to the JSON-array \enquote{Contains}, where each JSON-array in it has a pair where the first element corresponds to \enquote{Primary/Secondary Entity} and the second element corresponds to \enquote{Primary/Secondary Entity}.
\end{itemize}

\begin{figure}[h]
	\center
	\includegraphics[scale=0.5]{ecore_metamodel}
	\caption{Ecore meta-model inspired by Mosser et al. \cite{mosser2022modelling}}\label{fig:ecore_meta_model}
\end{figure} 
\subsubsection*{Creating Rules}\label{workflow_rule_creator}
With the identified USs in the the JSON-file, we generate rules with the Henshin package \textit{org.eclipse.emf.henshin.model.compact}, which is responsible for the creation of \textit{transformation rules} and their \textit{classes}, \textit{attributes}, \textit{edges} and annotates them with \textless\emph{Delete}\textgreater, \textless\textit{Create}\textgreater or\textless\textit{Preserve}\textgreater, which are crucial for the CDA tool to recognise the redundant pairs.
 
To generating rules we create a package named \textit{org.henshin.backlog.code.rule} and specially the class \textit{RuleCreator} which used following classes\footnote{https://wiki.eclipse.org/Henshin/Compact\_API}\label{compact_api}:
\begin{itemize}
\item \textit{org.eclipse.emf.henshin.model.compact.CModule}: CModule class can import elements from an Ecore file to use them in the transformation process responsible for linking the Ecore meta-model to the Henshin-file to be created.
\item \textit{org.eclipse.emf.henshin.model.compact.CRule}: Once we have a CModule, we can specify transformation rules with the CRule class and create them.
\item \textit{org.eclipse.emf.henshin.model.compact.CNode}: Now that we have a transformation rule, we want to fill this rule with nodes, edges and attributes. To create a node within a transformation rule, we need the CRule class. To create an edge we need to reference two nodes together. The default action when specifying a node, edge or an attributes is the \textless\emph{preserve}\textgreater action. We can also specify a different action when we create a node or an edge, for example \textless\emph{delete}\textgreater or\textless\emph{create}\textgreater.
\item org.henshin.backlog.code.rule.RuleCreator: We implement \textit{RuleCreator} class that creates a rule with annotated nodes, edges and attributes based on a JSON-file as input and a Henshin-file containing all rules as output, where each rule and its members(nodes, attributes, edges) correspond to the individual US and their JSON-objects/arrays in the JSON-file. 

The most important design decision of this class is the way nodes, attributes and edges are annotated in order to be able to apply conflict and dependency analysis (CDA) for rules stored as a henshin file.

We decided to annotate the \enquote{name} attribute of all Primary/Secondary Actions/Entities and their associated edges including \enquote{targets}, \enquote{triggers} and \enquote{contains} as \textless delete\textgreater action. 

The main goal is to increase the probability of identifying US-pairs characterised by matching names of \textit{action} nodes and \textit{entity} nodes in conjunction with an edge called \textit{targets}. This congruence serves as a basic criterion for identifying potentially redundant US-pairs and simplifies the process of redundancy detection in the context of US analysis.
\end{itemize}
\subsubsection*{Methods of the RuleCreator Class}
In this section, the methods of the RuleCreator class is described as follows:
\begin{enumerate}
	\item readJsonArrayFromFile: This method receives a JSON file as input. After reading, the JSON content is tokenised, parsed into a JSON array and the parsed JSON array is returned.
	\item assingCmodule: This method assign a CModule to a Ecore meta-model. It creates a new CModule object with the provided Henshin-file name, adds imports from the Ecore file, and returns the module.
	\item processJsonFile: It takes parsed JSON array as input and processes their attributes, such as persona, actions, entities, text and their edges, such as targets, triggers. Corresponding elements are created as output in a the Henshin transformation module (CModule).
	\item processRule: It takes the \enquote{US\_Nr} JSON-object as input and creates a new CRule with the name of unique US identifier in the CModule.
	\item processPersona: It receives as input the persona extracted from the JSON data and the associated CRule to create a new CNode representing the persona within the provided CRule and adds the attribute \enquote{name} with persona as value. Finally, the created CNode representing the persona is returned.
	\item processText: It receives as input US text extracted from JSON data and the associated CRule to create a new CNode representing the text within the provided CRule and adds the attribute \enquote{text} with US text as value. Finally, the created CNode representing the US text is returned.
	\item processActions: The processActions method is responsible for creating CNode objects that represent actions within the CModule. As parameters, it receives the JSON-object of the actions, the CNode-object representing the persona associated with the actions and the unique identifier of the US. Since the edge triggers only refer to the persona and the primary action, a new CNode is created for each primary action that represents the primary action within the provided CRule, an attribute \enquote{name} is added and an edge is created from the persona node to the primary action with the label \enquote{triggers}. For each secondary action, a new CNode is created to represent the action within the provided CRule and an attribute \enquote{name} is added to the action node.
	\item checkEntityIsTarget: It receives the name of the entity and the JSON-array with information about target edges. The method iterates through the JSON-array targets, which contains arrays that represent targets edges between actions and entities. It compares the targets entity with the specified entity. If there is a match, it returns true to indicate that the entity is a target.
	\item processEntities: It receives as parameters the JSON-object with information about the entities, the CRule object representing the US to which the entities belong and the JSON-array with information about the targets associated with the entities. The method checks whether primary/secondary entities are present, then creates a CNode for each primary/secondary entity and checks whether the entity is present in the target array. If this is the case, its attribute \enquote{name} is annotated for deletion. This method is used within the processContainsEdges method to determine whether an entity involved in a contains relationship is also a target of another entity.
	\item processContainsEdges: It receives the JSON-object to be processed, the JSON array with information about contains/target edges and the US identifier as parameters. It first checks whether both entities belong to contains edges. If both entities exist, an edge is created between them in CRule with the name \enquote{contains}. If one of the entities is a target of another entity (as specified in the targets array), the edge is annotated for deletion. If none of the entities is a target, the edge is annotated as \enquote{preserve}.
\end{enumerate}
Figure \ref{fig:rule_creator_methods} shows the sequence diagram of the RuleCreator methods.
\begin{figure}[h]
	\centering
	\includegraphics[scale=0.35]{RuleCreator_methods}
	\caption{Sequence diagram of the RuleCreator methods}\label{fig:rule_creator_methods}
\end{figure}
\subsubsection*{Conflicting and Dependency Analysis}
After the rules and the corresponding henshin file have been created by the RuleCreator class, we are now able to pass them to the conflict and dependency analysis (CDA) to find potential redundancy pairs.

Since the analysis of conflicts and dependencies related to the \textit{attribute} is not yet considered in the CPA API \footnote{https://wiki.eclipse.org/Henshin/conflict\_and\_Dependency\_Analysis}, we decided to use the user interface (UI) of the CPA extension of Henshin, which supports analysis of conflict and dependencies of rules through the interactive use of CPA.

To apply CDA to Henshin files, we just need to right-click on the Henshin file and select \textit{Henshin}\textgreater\textit{conflict and Dependency Analysis} from the context menu as shown in Figure \ref{fig:henshin_context_menu}.
\begin{figure}[h]
	\centering
	\includegraphics[scale=0.5]{henshin_context_menu}
	\caption{Applying CDA to the selected Henshin file}\label{fig:henshin_context_menu}
\end{figure}
A user interface then appears, prompting to select the rule sets to be analysed and the type of analysis. We then select as \enquote{\textit{First}} and \enquote{\textit{Second} \textit{Rules}}, all rules related to USs. Additionally, as the type of analysis we select \enquote{\textit{conflicts}} as illustrated in Figure \ref{fig:select_rules}.

\begin{figure}[h]
	\centering
	\includegraphics[scale=0.5]{select_rules}
	\caption{CDA user interface: Selection of rules and type of analysis}\label{fig:select_rules}
\end{figure}

 On the next page of the CDA UI shown in Figure \ref{fig:select_granularity}, we specify the depth of analysis that we use with \enquote{\textit{Fine granularity}} when selecting \enquote{\textit{Create a complete result table}} and \enquote{\textit{Create an abstract result table}}. We choose \enquote{Fine granularity} as the depth of analysis due to the fact that it shows all conflict reasons for each conflicting rule pair.
 
 A conflict reason is a model fragment whose presence leads to a conflict. General conflict reasons result from different combinations of minimal conflict reasons\cite{cda_api}.
 \begin{figure}[h]
 	\centering
 	\includegraphics[scale=0.5]{select_granularity}
 	\caption{CDA user interface: Selection of report granularity}\label{fig:select_granularity}
 \end{figure}
 
 During the execution of the CDA analysis, the rule pairs is analysed and a conflict analysis is performed. Once the calculation is complete, the results are listed in the \enquote{CDA} -\textgreater \enquote{Result window}, as shown in Figure \ref{fig:cda_report}. The top entry shows the granularity, which in our case is \enquote{Fine}. These entries contain the rule pairs that conflict with each other. Each rule pair contains a number of conflict reasons.
  \begin{figure}[h]
 	\centering
 	\includegraphics[scale=0.5]{cda_report}
 	\caption{CDA report with fine granularity}\label{fig:cda_report}
 \end{figure}
 
 Figure \ref{fig:cda_report_in_project_dir} shows how the data is saved in the project tree view. The results directory is created in the directory containing the Henshin that was used for the analyses. The new folder name is the date and time at which the analysis was performed. In contrast to the \enquote{\textit{CDA/Results}} view, this folder contains all conflict reasons and atoms together in a rule pair directory.
  \begin{figure}[h]
 	\centering
 	\includegraphics[scale=0.5]{cda_report_in_project_dir}
 	\caption{Saving CDA results data in the project structure view}\label{fig:cda_report_in_project_dir}
 \end{figure}
 
 For each conflict reason, there is a \enquote{\textit{minimal-model.ecore}} file, that contains packages in which various conflict elements such as \enquote{attributes} and \enquote{references} (edges) are mapped together and displayed in different packages.
 
 Figure \ref{fig:minimal_model_packages} shows the representation of the conflicting attributes and references. An attribute has the property of changing the value and is represented by an arrow \enquote{-\textgreater}. The attribute from the first rule is separated from the second rule by an underscore, just as with the nodes.
  \begin{figure}[h]
 	\centering
 	\includegraphics[scale=0.5]{minimal_model_packages}
 	\caption{Representation of conflicting attributes and references in \textit{minimal-model.ecore} file}\label{fig:minimal_model_packages}
 \end{figure}
\subsubsection*{Extracting Textual Report}
To create a lightweight report for the group or individual in question, we need to extract the key information from the CDA report, e.g. redundancy US-pair, redundancy clauses, count of redundancy clauses in each part of the US (main or benefit part), and create a report as a text file with the following information:
\begin{itemize}
	\item A table of potential redundant pairs with the number of total redundancy clauses.
	\item Founded potential redundant US-pairs.
	\item Redundancy words and clauses of founded US-pairs. Clauses consisting of two words that have one of the relationships triggers, targets or contains.
	\item Text of US-pairs whose redundancy words are marked with a hash symbol (\#).
	\item Parts of the sentence in which words and clauses are found.
\end{itemize}
Listing \ref{list:textual_report_sample} illustrates the example of the textual report. In this case, the report only contains one US-pair.

\begin{MyListing}
\paragraph{}
\hrule
\centering
\lstinputlisting[basicstyle=\ttfamily\footnotesize]{Listing/TextualReportSample.txt}
\caption{Example of generated textual report for one US-pair}\label{list:textual_report_sample}
\hrule
\end{MyListing}
In order to extracting a textual report associated with a specific backlog, we implement a class called \textit{ReportExtractor} within the package \textit{org.henshin.backlog.code.report}, which include the following classes form the package \textit{org.eclipse.emf.ecore}\footnote{https://download.eclipse.org/modeling/emf/emf/javadoc/2.7.0/org/eclipse/emf/ecore/}. These classes are important for reading the content of minimal-model.ecore:
\begin{itemize}
	\item org.eclipse.emf.ecore.resource.Resource: A resource of an appropriate type is created by a resource factory; a resource set indirectly creates a resource using such a factory. A resource is typically contained by a resource set, along with related resources.
	\item org.eclipse.emf.ecore.resource.ResourceSet: A resource set manages a collection of related resources and notifies about changes to this collection. It provides a tree of content. A collection of adapter factories supports the search for an adapter via a registered adapter factory. 
	\item org.eclipse.emf.ecore.EObject: EObject is the root of all modelled objects, therefore all method names start with "E" to distinguish the EMF methods from the client methods. It provides support for the behaviour and functions that are common to all modelled objects.
	\item org.eclipse.emf.ecore.EPackage: A representation of the model object \enquote{EPackage}.
	\item org.eclipse.emf.ecore.EClassifier: A representation of the model object \enquote{EClassifier}.
	\item org.eclipse.emf.ecore.EClass: A representation of the model object \enquote{EClass}.
	\item org.eclipse.emf.ecore.EAttribute: A representation of the model object \enquote{EAttribute}.
	\item org.eclipse.emf.ecore.EReference:  A representation of the model object \enquote{EReference}.
	
\end{itemize}
\subsubsection*{Storing Redundancy Items into RedundancyItems Class}
The following classes were created to represent the extracted model object accordingly. All these classes are extensions of the class \textit{RedundancyItems}, which contains all extracted model object from \enquote{minimal-model.ecore} such as \enquote{EClass}, \enquote{EAttribute} or \enquote{EReference}:
\begin{itemize}
	\item PrimaryAction/SecondaryAction: Which has only saved the EClass specified by \enquote{Primary/Secondary Action} and the EAttribute model object with the methods \textit{getType} to retrieve the saved EClass and \textit{getName} to retrieve the saved EAttribute.
		\item PrimaryEntity/SecondaryEntity: Which has only saved the EClass specified by \enquote{Primary/Secondary Entity} and the EAttribute model object with the methods \textit{getType} to retrieve the saved EClass and \textit{getName} to retrieve the saved EAttribute.
		\item Targets: The EClass specified by \enquote{Primary/Secondary Action} as \textit{outgoing edge} and an EAttribute model object as \textit{incoming edge} with \enquote{Primary/Secondary Entity}. The method \textit{getType} retrieve the stored EClass and method \textit{getName} retrieve the stored EAttribute.
		\item Contains: The EClass specified by \enquote{Primary/Secondary Entity} as \textit{outgoing edge} and an EAttribute model object as \textit{incoming edge} with \enquote{Primary/Secondary Entity}. The method \textit{getType} retrieve the stored EClass and method \textit{getName} retrieve the stored EAttribute.
		\item Triggers: The EClass specified by \enquote{Persona} as \textit{outgoing edge} and an EAttribute model object as \textit{incoming edge} with \enquote{Primary Action}. The method \textit{getType} retrieve the stored EClass and method \textit{getName} retrieve the stored EAttribute.
		\item RedundantPair: Stores the identifier of the two USs that were founded as a redundant pair. It also stored the total count of redundancy clauses within the US-pair.
		\item TargetsPair: Stores effective value of \enquote{\textit{Primary/Secondary Action}} and \enquote{\textit{Primary/Secondary Entity}} in action and entity fields accordingly.
		\item ContainsPair: Stores effective value of \enquote{\textit{Primary/Secondary Entity}} as \enquote{\textit{parent/child entity}} due to the fact that parent entity is a containment of child entity.
		\item TriggersPair: Stores effective value of \enquote{\textit{Persona}} as a persona and \enquote{\textit{Primary Action}} as an action.
\end{itemize}
The class RedundancyItems contains following methods which are crucial for class ReportExtractor:
\begin{itemize}
	\item isInCommonContains: This method is designed to determine whether a given entity (specified by its name) is part of a redundant pair listed in the \enquote{Contains} array of related USs stored in a JSON file.
	
	As input, it receives the name of the entity for which we want to check whether it exists in a redundant pair. In addition, a list of redundant pair objects that represent pairs of entities where one entity contains the other. If there is a match with the parent entity, it returns the child entity and vice versa.
	\item isInCommonTargets: This method is responsible for determining whether a particular action/entity is part of a redundant pair listed in the "Targets" array of related USs stored in a JSON file.
	
	As input, it receives the name and type of the first element, which is an action, and the name and type of the second element, which is an entity. It also receives a list of TargetsPair objects, which are pairs of common actions and entities between US-pairs.
	
	For each pair, it checks whether the specified names and types match either the action or entity in the pair. If a match is found, the method returns \textit{true}, which means that the specified elements are part of a redundant pair.
	\item printRedundantItems: This method is responsible for creating a report on redundancy items based on the data stored in the class instance. 
	
	As input, it takes several parameters, including a "FileWriter" to write to the textual report's file, lists of different pairs of "Targets", "Contains" and "Triggers", and a JSON-object to store the report data in JSON format, which is crucial for evaluation.
\end{itemize}
\subsubsection*{ReportExtractor Class: Methods related to Extracting Report}
In this subsection, the method of the ReportExtractor class is described as follows:
\begin{enumerate}
	\item extractReports: This method orchestrates the extraction and analysis of created CDA report from a directory containing conflicted US-pairs and their associated reasons(conflict reason), and generates both text and JSON reports for further investigation and processing.
	
	It receives two FileWriter objects as input, one for writing textual and one for JSON reports.
	
	It iterates through each directory in the main directory that represents a conflict pair and uses the \textit{checkIfReportExist} method to make sure that the current US-pair are not already proceeded and the \textit{containsAnd} method to check whether it contains the conjunction \enquote{AND} to make sure that it is the valid US-pair name(\textit{e.g.} "user\_stroy\_12\_AND\_user\_stroy\_39" is a valid directory name).
	
	For each valid conflict pair directory, it iterates through the conflict reason directories within the current conflict pair directory and uses the \textit{minimalEcoreExist} method to check whether a \textit{minimal-model.ecore} file exists with reference to a conflict reason.
	
	To identify redundancy items, it uses the method \textit{processMinimalModels} and reads the "minimal-model.ecore" file, processes its content and iterates over the contained EPackages in order to process them further with the method \textit{iteratePackages}, which saves all redundancy items in RedundacyItems object.
	
	In addition, the methods \textit{hasEntitys}, \textit{hasActions} and \textit{hasTargets} are used to check whether the identified elements contain "Primary/Secondary actions" and "Primary/Secondary Entities" with common "Targets" reference. If the identified elements fulfil the criteria, the redundancy pair is included in the report.
	
	It then writes the potentially redundant USs and their clauses to the textual as well as JSON report files for further analysis.
	
	As output, the method returns a list of RedundantPair objects containing information about identified redundancies between US-pairs.
	
	\item createOrOverwriteReportFile: The method is responsible for creating or overwriting report file. It first ensures the existence of a report file. If the file doesn't exist, it creates a new one; if it already exists, it overwrites the existing file. Finally, it returns a \textit{FileWriter} to allow writing to the report file.
	
	\item checkIfReportExist: This method takes two parameters, namely US-pair and the list of all previously processed pairs in the CDA report directory. It returns \textit{true} if the US-pair was found in the pairList, which means that a report with the specified pairs has already been executed and therefore does not need to be executed again.
	
	\item minimalEcoreExist: This method checks the existence of a \textit{minimal-model.ecore} file using a conflict pair and a conflict reason generated by CDA tool.
	
	As input, it receives a conflict pair and a conflict reason. Using these parameters, the method constructs the file path to the minimal-model.ecore file and checks whether the file exists under the constructed path.
	
	If the ECore file exists, the method returns \textit{true}, indicating that the minimal-model.ecore file exists for the examined conflict pair and conflict reason. If the file does not exist, \textit{false} is returned.
	
	\item containsAnd: This method ensures that the folder name is identified with "\textit{\_AND\_}", as the report generated by CDA tool is formatted like \enquote{user\_story\_\textless digit \textgreater \_AND\_ user\_story\textless digit \textgreater}. It returns \textit{true} if the examined directory contains \enquote{AND}, otherwise \textit{false} is returned.
	
	\item processMinimalModels: This method reads a Minimal Model Ecore file, processes its content and iterates over the contained EPackages in order to process them further with the \textit{iteratePackages} method. 
	
	As parameters, it receives a File object that represents examined minimal model ecore file, an array list in which the names of the redundant elements are stored, and a RedundancyItems object that is used to handle redundant elements. 
	
	First, a ResourceSet and a ResourceFactoryRegistry corresponding to the minimal model Ecore file are set up and a Resource object is created from the Ecore file; the \textit{iteratePackages} method is called for each EPackage.
	
	\item iteratePackages: This method identifies redundancies within the attributes and references of EClasses in examined minimal model, stored them accordingly and updates the RedundancyItems object.
	
	It takes several parameters, such as the EPackage to be iterated over, an array list in which the names of the redundant elements are stored, and a RedundancyItems object. 
	
	It iterates through every EClassifier in the minimal package that contains EClasses and checks whether \enquote{\#} is present in EClass; if this is the case, EClass has been recognised as a conflict by CDA tool.
	
	If an attribute is found, the class of the conflicting attribute is determined and added to the corresponding element within RedundancyItems (e.g. Primary/Secondary Action/Entity). 
	
	Each EReference is then iterate through in the EClass. Depending on the reference name, the reference is added to the corresponding element within RedundancyItems (e.g. Triggers, Targets, Contains). The method is completed once all EClassifiers within the specified EPackage have been processed.
	
	\item hasActions: This method is useful for using the data stored in RedundancyItem to determine whether certain actions are present in a list of redundant entries, in order to later check whether the identified elements fulfil the criteria and the redundancy pair is included in the report.
	
	It checks if there's a match with name of any primary/secondary action stored in the RedundancyItems object. If match is found, it immediately returns \textit{true}.
	
	\item hasEntitys: This method is used to determine whether certain secondary/primary entities  are present in the RedundancyItems object based on their name. For each item, it checks whether there is a match with the name of a primary/secondary entity stored in the RedundancyItems object. 
	
	If a match is found, \textit{true} is returned immediately, in order to later check whether the identified elements fulfil the criteria and the redundancy pair is included in the report.
	
	\item hasTargets: This method uses the content of a RedundancyItems object to determine whether targets are present in a list of founded redundant elements.
	
	As input, it receives an array list with the names of the redundant elements and an object of the type RedundancyItems, which contains a collection of founded "Targets".
	
	The method checks whether there is a match with the name of any targets reference stored in the RedundancyItems object. 
	
	If a match is found between the elements, the method immediately returns \textit{true}, which means that at least one targets reference is present, in order to later check whether the identified elements fulfil the criteria and the redundancy pair is included in the report.
	
	\item readJsonArrayFromFile: This method provides the ability to read JSON data from a file and convert it into a JSONArray object, handling cases where the file is empty or does not exist.

	It receives the file path of the JSON file to be read as input. An attempt is made to open the specified file and check whether the file is empty or does not exist. If the file exists and is not empty, it reads the JSON data from the file and creates a JSON array object from the JSON data read from the file and returns the JSON array object with the JSON data.
	
	\item getUssTexts: This method ensures that the text of the specified US-pair is retrieved from the JSON file and properly assigned to the RedundancyItems object for further processing. It receives a US-pair and RedundancyItems as input. 
	
	It reads a JSON array from a file using the \textit{readJsonArrayFromFile} method, iterates over each JSON object in the array and compares the extracted US identifier with the US identifier extracted from the input US-pair. If a match is found, the text of the first and second USs is set in the redundancyItems object.
	\item getRedundancyStatus: This method add statistics like count of main/benefit/total redundancies into JSON report. It receive as input redundancyItems and as output write the count of main/benefit/total redundancies into JSON report already defined in method highlightRedundancies.
\end{enumerate}
\subsubsection*{Methods for Highlighting Words using Hash Symbol(\#)}
In order to distinguish redundancy words between a US-pair, in text of each US, we decide to highlighting redundant words using hash symbol like \enquote{ ... \#\textless word\textgreater\# ... }.

The following methods in the ReprortExtractor class are responsible for highlighting redundancy words:
\begin{enumerate}
	\item writeUsText: This method reads the text of examined US-pair, highlights redundant element between them using \textit{highlightRedundancies} method, writes the highlighted text to a file, and records information about the redundant elements. 
	
	As input it receive a US-pair, a list of redundant pairs to store redundant pairs in it, an object RedundancyItems containing stored redundancy elements and a FileWriter object used for writing output.
	
	It extract the identifier of the two USs from US-pair, retrieves the text of the USs from JSON file and add them to RedundancyItems, invoke the \textit{highlightRedundancies} method to identify and highlight redundants between the USs, it writes the highlighted text of each US to the FileWriter. Finally, sets the redundant pair and count of redundancy clauses.
	\item highlightRedundancies: This method identifies redundancies between US-pair, applies hash symbols using \textit{applyHashSymbols} method to highlight common items and updates the redundancy counts in the redundancyItems object. 
	
	It takes two parameters, redundancyItems and US-pair, which represents the pair of USs to be analysed. 
	
	It checks whether both USs contain a main clause part or whether one of them has a benefit part or whether both USs also have a benefit part.
	
	It applies hash symbols to common elements that only occur in the part of the USs that occurs in the same part (e.g. only main or only benefit part of the USs). 
	
	In each condition, it checks if there are redundancy clauses in the main part, then persona is also highlighted using \textit{applyHashSymbolPersona} method. It also updates the count of main/benefit/total redundancies and sets the changed text of USs. Finally, it returns the updated redundancyItems object.
	\item applyHashSymbols: This method is used to mark certain words within a substring with hash symbols (\#) at the beginning and end to ensure that they are distinguishable and can be easily identified or processed later. 
	
	It takes a substring in which replacements are to be made and a field of matches containing the words to be surrounded with hash symbols. First, the field of matches is sorted in descending order of length and processed accordingly to avoid adding hash symbols to unwanted clauses. 
	\begin{example}
		For example, let's assume that we have \enquote{data} and \enquote{data format} as redundancy elements. If we continue first with \enquote{data} and then with \enquote{import data}, \enquote{import data} will be replaced by \enquote{import \#data\#}, which is not desired.
	\end{example}
	\item hasMoreThanFour/SixHashSymbols: These methods receive a text from the US as input. They are used to check whether there are redundant clauses in the main part of the sentence (it can be one or two clauses). If so, \textit{true} is returned.
	
	\item applyHashSymbolPersona: This method identifies common "Triggers" references, marks them with hash symbols and returns the changed text parts together with the count of redundant triggers references. 
	
	As input, it receives a list of common triggers between US-pair, RedundancyItems and the parts of the USs. It iterates through the list of common triggers and checks whether both elements(persona and primary action) of the triggers are present in both parts. 
	
	It then increments the redundancy count to keep track of the count of redundant triggers. The output returned, is the text of USs containing the manipulated text parts with hash symbols and the redundancy count in main part.
	
	\item applyHashSymbolTargets: This method identifies common targets references between two parts of USs, marks them with hash symbols and returns the changed text parts together with the count of redundant targets references.
	
	As input, it receives a list of common targets between the US-pair, RedundancyItems and the parts of the USs. 
	
	It iterates through the list of common targets and checks whether both elements (primary/secondary action and primary/secondary entity) of the targets are present. It then increments the redundancy count in examined part of USs to keep track of the count of redundant targets. 
	
	The output returned is the text of USs containing the manipulated text parts with hash symbols and the redundancy count in examined part(main or benefit).
	
	\item applyHashSymbolContaians: This method identifies common contains references between two parts of USs, marks them with hash symbols and returns the changed text parts together with the count of redundant contains references. 
	
	As input, it receives a list of common contains between the US-pair, RedundancyItems and the parts of the USs. It iterates through the list of common contains and checks whether both elements of the contains are present in both parts. It then increments the redundancy count to keep track of the count of redundant contains.
	
	The output returned is the text of USs containing the manipulated text parts with hash symbols and the redundancy count in examined part(main or benefit).
	
\end{enumerate}
\subsubsection*{Methods related to Report Parts of USs}
To show which part of the USs with redundancy words occur between a US-pair, we have split the individual parts of the USs and included them in the report:

The following methods in the ReprortExtractor class are responsible for splitting and reporting the parts of the USs :
\begin{enumerate}
	
	\item splitUsText: This method is used to split the text of two USs into separate sections based on the occurrence of redundancy clauses. 
	
	The input is the text of the first and second US and their corresponding identifiers, a FileWriter for writing to a file and a JSON object for processing JSON data. 
	
	It splits each US text into three parts using commas and saves the result in arrays. It iterates over parts of the first and second USs and searches for occurrences of hash symbol pairs. For each part, the number of hash symbol pairs found is counted. Finally, all parts of the records that contain hash symbols are written to a text file and a JSON file as well.
	
	\item writeUsSentencePart: This method facilitates the extraction and storage of highlighted USs parts from US-pair in a textual report file for further analysis.
	
	As input, it receives the US-pair, RedundancyItems and a FileWriter allowing the extracted USs parts to be written. It receives the text of the USs from the reundancyItems object and calls the \textit{splitUsText} method to split the US-pair texts into USs parts with highlighted elements.
	
	The extracted USs parts are also written to the text and JSON report files, using the FileWriter for further processing and analysis.
	
	\begin{example}
		Take, for example, the following US-pair:
		
		\textit{user\_story\_60:} \#g22\# as an it staff member, I want to \#know\# \#how\# the \#data\# is \#used\#, so that I can determine what kind of basic services and functionalities are required.\\\\
		\textit{user\_story\_04:} \#g22\# as a data manager, I want to \#know\# \#how\# the \#data\# is \#used\#, so that I can develop more detailed usage and support scenarios with researchers.\\\\
		The following sentence parts are candidates for possible redundancies between user stories:\\\\
		user\_story\_04:  I want to \#know\# \#how\# the \#data\# is \#used\#\\\\
		user\_story\_60:  I want to \#know\# \#how\# the \#data\# is \#used\#\\\\
		
	\end{example}
\end{enumerate}
\subsubsection*{Methods related to Creating Table}
The following methods in the ReprortExtractor class are responsible for creating a table at the beginning of the text report:
\begin{enumerate}
	\item writeTable: This method is used to write a table of potential redundancies between USs and the count of their total redundant clauses \textit{createTable} method.%, including the addition of the count of main and benefit redundancy clauses using \textit{createTable} method.
	
	As input, it receives a File object into which the table is inserted and a list of redundancy pairs containing information about redundant clauses in US-pars.
	
	It reads the existing content of the textual report file into a StringBuilder. It creates a table to display the potential redundancies between USs and the count of total redundancy clauses.
	
	The table headers and contents are generated based on the redundant pairs. It calculates the maximum width for each column in the table to ensure proper formatting. Finally, the table content is written to the FileWriter, followed by the existing content stored in the report's StringBuilder.
	
	\item createTable: This method prepares the content for the table in which potential redundancies between USs are displayed, taking into account the total redundancy count between each US-pairs based on the RedundantPair objects provided.
	
	As input, it receives a list of unique US-pairs for which the table is to be created and a list of RedundantPair objects containing information about redundant clauses between US-pairs.
	
	It initialises a two-dimensional array containing the contents of the table. The size of the table is determined by the count of unique pairs of USs plus one for the header row and column.
	
	It fills the header row and the first column of the table with unique pairs of US-pairs, replacing \enquote{user\_story} with \enquote{us} for the purpose of brevity.
	
	It calculates the maximum redundancy count between each US-pair by calling the method \textit{getTotalRedundanciesFromPair}. Finally, it fills the table with the total redundancy count. 
	
	The output is a two-dimensional array representing the contents of the table, with each cell containing the maximum redundancy count between the corresponding pair of USs.
	
	\item getTotalRedundanciesFromPair: This method makes it easier to retrieve the total number of redundancies between a US-pair from a list of RedundantPair objects.
	
	As input, it receives a list of RedundantPair objects containing information about redundant elements in US-pairs, where the first and second USs are to be compared.
	
	It iterates through each RedundantPair object in the RedundantPairs list and checks for each RedundantPair object whether the examined US-pair matches as redundant US-pair. 
	
	If a matching pair is found, the maximum redundancy number stored in this RedundantPair object is returned. If no matching pair is found, \textit{zero} is returned, indicating that there are no redundancies between the examined US-pair.
\end{enumerate}
\subsection{Implementation}\label{redundancy_implementation}
In this section, we explain the objective and scope of the implementation, the system architecture, the functionality and the programming languages used.

The entire implementation is available in the GitHub repository \footnote{https://github.com/amirrabieyannejad/USs\_Annotation/tree/main}.
\subsubsection*{Objective and Scope}
The goal and scope of the work is divided into three phases. Firstly, converting the USs annotated by the CRF tool into graph transformation rules; secondly, using the CDA tool of the Henshin to automatically report redundancies between USs pairwise; thirdly, extracting important information from the CDA report into a text report. For further analysis, we stored the information in a JSON file to be able to import the data into another platform such as Excel.

Figure \ref{fig:implementation_phases} illustrates the mentioned implementation phases.
\begin{figure}[h]
	\centering
	\includegraphics[scale=0.3]{implementation_phases}
	%\includegraphics[scale=0.35]{sequence_diagram}
	\caption{Three Implementation phases}\label{fig:implementation_phases}
\end{figure} 
\subsubsection*{Methodology}
This section explains and introduces tools that are required during the development process.

Following approach and tools are necessary in order to develop our workflow:
\begin{itemize}
	\item Eclipse as IDE\footnote{https://eclipseide.org/}: Eclipse is an integrated development environment (IDE) used in computer programming. It contains a base work workspace and an extensible plug-in system for customizing the environment.
	
	\item Eclipse Modeling Project\footnote{https://eclipse.dev/modeling/}: It focuses on the evolution and promotion of model-based development technologies within the Eclipse community by providing a unified set of modelling frameworks, tooling, and standards implementations.
	
	\item Eclipse Modeling Framework (EMF)\footnote{https://eclipse.dev/modeling/emf/}: The EMF project is a modeling framework and code generation facility for building tools and other applications based on a structured data model. From a model specification described in XMI, EMF provides tools and runtime support to produce a set of Java classes for the model, along with a set of adapter classes that enable viewing and command-based editing of the model, and a basic editor.
	
	\item Henshin\footnote{https://wiki.eclipse.org/Henshin}: Henshin is an in-place model transformation language for the Eclipse Modeling Framework (EMF). It supports direct transformations of EMF model instances (endogenous transformations), as well as generating instances of a target language from given instances of a source language (exogenous transformations).
	
	\item Henshin's CDA feature\footnote{https://wiki.eclipse.org/Henshin/Conflict\_and\_Dependency\_Analysis}: Henshin's conflict and dependency analysis (CDA) feature enables the detection of potential conflicts and dependencies of a set of rules.
	
	\item Java as programming language\footnote{https://www.java.com/de/}: Java is a widely used object-oriented programming language and software platform that is used to implement the Henshin and EMF APIs, which are critical to our approach to utilising them. Therefore, we use Java as our programming language.
	
	\item GitHub as version control\footnote{https://github.com/}: GitHub is a developer platform that allows developers to create, store, manage and share their code. It uses Git software, providing the distributed version control of Git plus access control, bug tracking, software feature requests, task management, continuous integration, and wikis for every project.
\end{itemize} 
\subsubsection*{Data Structures}
The datasets of the backlogs annotated with the CRF tool have the JSON format, the structure of which is very important in the Java classes \textit{RuleCreator} and \textit{ReportExtractor}. Therefore, understanding the JSON format provided is crucial for the further procedure.

Each JSON file for a backlog dataset contains a JSON-array in which each US entry is defined as a JOSN-object. Listing \ref{list:textual_report_sample} illustrates the format used for the US entry.
\begin{MyListing}
	\paragraph{}
	\hrule
	\centering
	\lstinputlisting[basicstyle=\ttfamily\footnotesize]{Listing/json_format.json}
	\caption{The JSON format of each US entry in JSON file}\label{list:textual_report_sample}
	\hrule
\end{MyListing}
To interact with the entries in JSON file, we need to distinguish between the entries that are defined as JSON-objects, such as: {Text, Action, Entity, Benefit, US\_Nr} and the entries that are defined as a JSON-array, such as: {Persona, Primary/Secondary Action, Primary/Secondary Entity, Triggers, Targets, Contains}.

In order to parsing JSON file, we use \enquote{org.json} library which provide us the classes as follow:
\begin{itemize}
	\item \enquote{JSONObject}: An unordered collection of key and value pairs.
	\item \enquote{JSONArray}: Provide an ordered sequence of values.
	 \item \enquote{JSONTokener}: A tool that breaks a piece of text into a series of tokens that can be used by JSONObject or JSONArray to parse JSON entries.
\end{itemize}
%Provide detailed explanations of key implementation aspects, such as algorithms, data structures, design patterns, and any custom solutions developed. Discuss any challenges faced during implementation and how you addressed them.
\subsubsection*{Error Handling}
There were erroneous data in datasets that force us to handle them correctly. Therefore, we implement/use the following exceptions to accurately distinguish and handle them.

The following classes, which extend the \textit{Exception} class, are used in the \textit{org.henshin .backlog.code.rule} package and relate to error handling related to the JSON entries in the dataset of the backlogs and to the Ecore meta-model required to create the rules based on it:
\begin{itemize}
	\item EmptyOrNotExistJsonFile: Is triggered if the JSON file could not be found in the file system.
	
	\item ActionInJsonFileNotFound: Is triggered if the entry \textit{Action}, which contains \textit{Primary/Secondary Actions}, is not present in the JSON file and its absence should be reported.
	
	\item EntityInJsonFileNotFound: Is triggered if the entry \textit{Entity}, which contains \textit{Primary/Secondary Entity}, is not present in the JSON file and its absence should be reported.
	
	\item PersonaInJsonFileNotFound: Is triggered if the entry \textit{Persona} does not exist in the JSON file for a specific US.
	
	\item TextInJsonFileNotFound: Is triggered if the entry \textit{Text} does not exist in the JSON file for a specific US. 
	
	\item UsNrInJsonFileNotFound: Is triggered if the entry \textit{Us\_Nr} does not exist in the JSON file for a specific US.

	\item EdgeWithSameSourceAndTarget: Refers to the creation of edges in graph transformation rules and is triggered if the source and target of the edge have already been created, then the duplicate edge should be avoided.
		
	\item TargetsInJsonFileNotFound: Is triggered if the entry \textit{Targets} does not exist in the JSON file for a specific US.
	
	\item ContainsInJsonFileNotFound: Is triggered if the entry in \textit{Contains} is not present as \textit{Primary/Secondary Entity} in the JSON file and its absence should be reported.
	
	\item TriggersInJsonFileNotFound: Is triggered if the entry \textit{Triggers} does not exist in the JSON file for a specific US. 
	
	\item EcoreFileNotFound: Is triggered if the required ECore meta-model file could not be found and should also be reported.
\end{itemize}
Within the \textit{org.henshin.backlog.code.report} package, special classes that extend the \textit{Exception} class are designed to solve problems related to the CDA report directory, which encapsulates all US-pairs together with the associated conflict reasons. These classes include:
\begin{itemize}
	\item CdaReportDirIsEmpty: This exception is called if the CDA report directory is found but has no content.
	\item CdaReportDirIsNotADirectory: This exception is thrown in scenarios where the path provided for the CDA report directory is either not a directory (e.g. it is a file) or the specified path does not lead to a directory.
	\item CdaReportDirNotFound: This exception is triggered if the CDA report directory cannot be found within the specified path.
\end{itemize}
\subsubsection*{Structure of a US}
A US is divided into two different parts:
\begin{itemize}
	
	\item The \textit{main part}, which encapsulates the persona, the intended functionality and the resources required to execute the action.
	
	\item The \textit{benefit part}, which formulates the potential benefit for the end user and typically begins with the phrase \enquote{, so that}.
	
\end{itemize}
In the implementation phase, the delineation and checking of redundancy clauses within USs requires a methodical approach, especially when distinguishing between the main and the benefit part of a US-pair. This distinction is crucial to ensure that redundancy identifications within one part are not mistakenly transferred to the other. Consequently, the analytical framework comprises three conditions, each of which specifies its own methodology for case processing:
\begin{itemize}
	\item Presence of benefit in both USs of the redundancy-pair: if a benefit is identifiable in each US of the pair, a process of separation is used to split the main content from the benefit parts. After this separation, a targeted search for redundancy clauses is carried out only within the main part, whereby identified redundancies are annotated with a hash symbol (\#). This process is repeated for the benefit parts to ensure a thorough check and marking of redundancies within each individual part.
	
	\item Exclusive presence of a benefit in one US of the redundancy-pair: In scenarios where only one US of the pair contains a benefit part, the analysis is limited to the main part of both USs. The aim remains the identification and annotation of redundancy clauses within this part. The lone benefit part remains in its original state and is excluded from the redundancy check.

	\item Absence of benefit parts in both USs of the pair: If neither of the two USs of the redundancy-pair contains a benefit part, the focus shifts completely to the main parts. The investigation is designed to highlight redundancy clauses within these parts, whereby the benefit parts are not taken into account due to their non-existence.
	
\end{itemize}
This structured and segmented approach ensures precise and efficient identification of redundancy clauses within the USs, optimising the clarity and effectiveness of textual report.

\subsubsection*{Limitations}
There are technical limitations that are causing us to change our implementation strategy.
The following limitations should be clarified at the beginning:
\begin{itemize}
	\item Use Eclipse version 2023-03, as Henshin version 4 cannot be installed with the latest version of Eclipse.
	
	\item Working with Java as the programming language, as the Henshin and CDA APIs are only available in the Java programming language.
	
	\item CDA API is not yet implemented to take into account conflicts and dependencies for attributes that are crucial for our approach. This forces us to use the CDA graphical user interface(GUI) instead of the CDA API.
	
	\item Lack of Henshin documentation regarding methods and classes, which makes it time consuming to understand the methods and make the right decision.
\end{itemize}
\subsection{Test}\label{redundancy_test}
In this section, we aim to validate certain functionalities, check the system requirements and ensure reliability and robustness of implemented classes and methods. As a test strategy, we perform unit tests with \textit{JUnit} version 4\footnote{https://junit.org/junit4/} as version 4 is more suitable and compatible with Eclipse version 2023-03.

We also used EclEmma\footnote{https://www.eclemma.org/}, a code coverage tool integrated into the Eclipse IDE, to ensure thorough testing and coverage. By using EclEmma, we were able to systematically measure the effectiveness of our test suites and determine the test coverage for each individual class.
\subsubsection*{Configuration of Test Environment}
In the main project \textit{org.henshin.backlog}, we create a separate package called \textit{org.henshin. backlog.test}. This package contains two Java classes namely \textit{ReportExtractorTest.java} and \textit{RuleCreator\_Test.java}, each of which corresponds to the Java source codes accordingly.
\subsubsection*{Scope of Testing}
The scope of the test depends on the system requirements and the two most important implemented classes \textit{ReportExtractorTest.java}, \textit{RuleCreator\_Test.java},\textit{Evaluation.java} and their methods. The implemented error handling classes are also tested.
\subsubsection*{Test Cases and their Code Coverage}
We describe the specific test cases that are performed during the tests. Each test case contains a description of the test scenario, the data provided and the expected result. For refining and improving our test cases, we used a code coverage report created by EclEmma in order to increase coverage and ensure a more reliable, error-resistant application.

Table \ref{tb:test_cases_rule_creator} shows the test cases for the RuleCreator.java class and Figure \ref{fig:code_coverage_rule_creator} shows the code coverage.

%\newgeometry{margin=2.5cm}
%\begin{landscape}
\thispagestyle{empty}
%\begin{figure}[h]
\begingroup
\centering
\scriptsize
\renewcommand{\arraystretch}{1,5} 
\keepXColumns
	\begin{tabularx}{\textwidth}{X  X  X  X}
	\hline
	Test Case &Supplied Data&Expected Outcome&Description\\
	\hline\hline
	\endfirsthead
	\hline
	Test Case &Supplied Data&Expected Outcome&Description\\
	\hline\hline
	\endhead
		testAssignCmodule&Assing a dummy ECore model file&Through an exception: \textit{EcoreFileNotFound}&Check whether the ECore model already exists and CModule is correctly assigned\\
		
		ActionNotExist (testProcessJsonFile)&JSON file with a US without \enquote{Action} entry&Through an exception:  \textit{ActionInJsonFileNotFound}&Check whether there is an entry \enquote{Action} in the related US in JSON file\\
		
		
		EntityNotExist (testProcessJsonFile)&JSON file with a US without \enquote{Entity} entry&Through an exception: \textit{EntityInJsonFileNotFound}&Check whether there is an entry \enquote{Entity} in the related US in JSON file\\
		
		PersonaNotExist (testProcessJsonFile)&JSON file with a US without \enquote{Persona} entry&Through an exception: \textit{PersonaInJsonFileNotFound}&Check whether there is an entry \enquote{Persona} in the related US in JSON file\\
		
		TargetsNotExist (testProcessJsonFile)&JSON file with a US without \enquote{Targets} entry&Through an exception: \textit{TargetsInJsonFileNotFound}&Check whether there is an entry \enquote{Targets} in the related US in JSON file\\
		
		ContainsNotExist (testProcessJsonFile)&JSON file with a US without \enquote{Contains} entry&Through an exception: \textit{ContainsInJsonFileNotFound}&Check whether there is an entry \enquote{Contains} in the related US in JSON file\\
		ContainsInTargets (testProcessContainsEdges)&JSON file with a US which the entity in \enquote{Contains} entry also belong to Targets&The entity in Contains should be annotated as \textless Delete\textgreater&Check whether entity in Targets have connection as Contains, if so entity in Contains should also annotated as \textless Delete\textgreater\\
		
		TriggersNotExist (testProcessJsonFile)&JSON file with a US without \enquote{Triggers} entry &Through an exception: \textit{TriggersInJsonFileNotFound}&Check whether there is an entry \enquote{Triggers} in the related US in JSON file\\
		
		DuplicateTriggers (testProcessContainsEdges)&JSON file with a US with duplicate Triggers entries&Through an exception:\textit{EdgeWithSameSource AndTarget}&Verify whether there is duplicated entries in Triggers JSON-array\\
		
		TextNotExist (testProcessJsonFile) &JSON file with a US without \enquote{Text} entry&Through an exception: \textit{TextInJsonFileNotFound}&Check whether there is an entry \enquote{Text} in the related US in JSON file\\
		
		UsNrNotExist (testProcessJsonFile)&JSON file with a US without \enquote{Us\_Nr} entry&Through an exception: \textit{UsNrInJsonFileNotFound}&Check whether there is an entry \enquote{US\_Nr} in the related US in JSON file\\
		
		UndefindedEntity \newline(testProcessContainsEdges)&Specify an entity that is not contained as \enquote{Entity} in the JSON file, but appears as \enquote{Contains}&Through an exception:  \textit{EntityInJsonFileNotFound}&Check whether the entity that appears in the \textit{Contains} entry has already been identified as an entity\\
		
		PrimaryActionNotFound \newline(testProcessTargetsEdges)&Specify a primary action that is not contained as \enquote{Primary Action} in the JSON file, but appears as \enquote{Targets}&Through an exception: \textit{ActionInJsonFileNotFound}&Check whether the primary action that appears in the \textit{Targets} entry has already been identified as a primary action\\
		
		PrimaryEntityNotFound \newline(testProcessTargetsEdges)&Specify a primary entity that is not contained as \enquote{Primary Entity} in the JSON file, but appears as \enquote{Targets}&Through an exception: \textit{EntityInJsonFileNotFound}&Check whether the primary entity that appears in the \textit{Targets} entry has already been identified as a primary entity\\
		
		SecondaryActionNotFound \newline(testProcessTargetsEdges)&Specify a secondary action that is not contained as \enquote{Secondary Action} in the JSON file, but appears as \enquote{Targets}&Through an exception: \textit{ActionInJsonFileNotFound.class}&Check whether the secondary action that appears in the \textit{Targets} entry has already been identified as a secondary action\\
		
		SecondaryEntityNotFound \newline(testProcessTargetsEdges)&Specify a secondary entity that is not contained as \enquote{Secondary Entity} in the JSON file, but appears as \enquote{Targets}&Through an exception: \textit{EntityInJsonFileNotFound.class}&Check whether the secondary entity that appears in the \textit{Targets} entry has already been identified as a secondary entity\\
		
		UndefindedEntity (testProcessTargetsEdges) &Specify an entity that is not contained as \enquote{Entity} in the JSON file, but appears as \enquote{Targets}&Through an exception:  \textit{EntityInJsonFileNotFound}&Check whether the entity that appears in the \textit{Targets} entry has already been identified as an entity\\
		
		ActionNotFound&Provision of a JSON file that contains actions in \textit{Targets} that are not entered in primary/secondary action&Through an exception:  \textit{EntityInJsonFileNotFound}&Verify whether the action that appear in the \textit{Targets} entry has already been identified as an action\\
		
		ReadJsonArrayFromFile&Assign a dummy JSON file&Through an exception:  \textit{EmptyOrNotExistJsonFile}&Check whether the JSON file already exists\\
		
		EmptyOrNotExistJsonFile&Assign an empty JSON file&Through an exception: \textit{EmptyOrNotExistJsonFile}&Check whether the JSON file is empty\\
		\hline
		\caption{Test cases for RuleCreator  class}\label{tb:test_cases_rule_creator}
	\end{tabularx}
	%\captionof{table}{Test cases for RuleCreator  class}\label{tb:test_cases_rule_creator}
\endgroup

\begin{figure}[h]
	\centering
	\includegraphics[scale=0.35]{code_coverage_rule_creator}
	%\includegraphics[scale=0.35]{sequence_diagram}
	\caption{Code coverage related to class RuleCreator}\label{fig:code_coverage_rule_creator}
\end{figure} 


Table \ref{tb:test_cases_report_extractor} shows the test cases for the ReportExtractor.java class and Figure \ref{fig:code_coverage_report_extractor} shows the code coverage.

%\end{figure}
%\end{landscape}
%\restoregeometry
%\newgeometry{margin=2.5cm}
%\begin{landscape}
	\thispagestyle{empty}
%	\begin{figure}[h]
		\begingroup
		\centering
		\scriptsize
		\renewcommand{\arraystretch}{1,5} 
		\keepXColumns
		\begin{tabularx}{\textwidth}{X  X  X  X}		
			\hline
			Test Case &Supplied Data&Expected Outcome&Description\\
			\hline\hline
			\endfirsthead
			\hline
			Test Case &Supplied Data&Expected Outcome&Description\\
			\hline\hline
			\endhead
			testEmptyDirectroy&Assing a dummy directory&Through an exception: \textit{CdaReportDirIsEmpty.class}&Check if CDA Report directory is empty\\
			
			testEmptyJSONFile&Assing an empty JSON dataset file&Through an exception: \textit{EmptyOrNotExistJsonFile}&Check if JSON dataset file is empty\\
			
			testCdaDirNotDirectroy&Assing a file instead of CDA directory&Through an exception:\textit{CdaReportDirIsNotA Directory}&Check if assigned path is a directory\\
			
			testCdaDirectroy&Assing a not readable directory&Through an exception:\textit{CdaReportDir NotFound}&Check whether CDA directory is accessible\\
			
			testFinalReportDir&Assign an empty dataset directory&Return \textit{false} if the assigned directory is empty&Check if directory of datasets are empty\\
			
			testMinimalEcoreExist&Assign a CDA directory without minimal-ecore file&Return \textit{false} if the minimal-ecore file not found&Check whether the Ecore file already exist in CDA directory\\
			
			testWriteTable&Provide a CDA report for a US-pair&A table should be created in the report file&Check whether the table for US-pairs has already been created\\
			
			completeMajorElements \newline Edge\newline(testExtractReports)&Provision of a CDA report for US-pair with exactly one \enquote{Targets} edge&Inclusion of this US-pair in the text report with the nodes \enquote{Action}, \enquote{Entity} and \enquote{Targets} edge&Verifies \textit{extractReports} method when all major elements are present in the CDA report\\
			
			completeMajorElements\newline Upper Edge\newline(testExtractReports)&Provide a CDA report for a US-pair with at least one \enquote{Targets} edge and redundancy elements such as Triggers&Generated textual report contains information about Secondary Entities, Secondary Actions, Targets, and Triggers&Verifies \textit{extractReports} method when all major elements are present in the CDA report and the upper edge case is reached\\
			
			notCompleteMajor\newline Elements\newline(testExtractReports)&Provide a CDA report for a US-pair without \enquote{Targets} edge, but with action and entity&The US-pair should not reported&Verifies the behavior of the \textit{extractReports} method when not all major elements are present in the input data\\
			
			getBenefitPart\newline RedundanciesElements\newline(testExtractReports)&Provision of a CDA report for a US-pair that only contains redundancy clauses in the \enquote{Benefit} part&Check whether the count of redundancy clauses in the benefit part of the USs matches the value \enquote{Benefit Part Redundancy Clause} specified in the
			JSON\_Report file&Verifies the behaviour of the \textit{extractReports} method when there are redundancy elements in the benefit part of USs\\
			
			getMainPartRedundancies\newline Elements\newline(testExtractReports)&Provision of a CDA report for a US-pair that only contains redundancy clauses in the \enquote{Main} part&Check whether the count of redundancy clauses in the main part of the USs matches the value \enquote{Main Part Redundancy Clause} specified in the JSON\_Report file&Verifies the behaviour of the \textit{extractReports} method when there are redundancy elements only in the main part of USs\\
			
			getTotalRedundancy\newline Elements\newline(testExtractReports)&Provision of a CDA report for a US-pair that contains redundancy clauses in the \enquote{Main}and \enquote{Benefit} parts&Check whether the count of redundancy clauses in the main and benefit parts of the USs matches the value \enquote{Total Redundancy Clause} specified in the JSON\_Report file&Verifies the behaviour of the \textit{extractReports} method when there are redundancy elements in the main and benefit parts of USs\\
			
			highlightPersona\newline(testExtractReports)&Providing a CDA report for a US-pair with redundancy clause in \enquote{Triggers}edge (from Persona to Primary Action)&The persona should only be marked with hash symbol if there is a redundant clause in the main part&Checks the behaviour of the \textit{extractReports} method when highlighting redundant persona in USs\\
				
			BenefitInBothUss\newline(testhigHlightRedundancies)&Provision of a CDA report for a US-pair where both USs have the \enquote{benefit} part&Search both the main and the benefit parts of USs and mark redundancy clauses with a hash symbol if they occur&Verifies the behaviour of the \textit{highlightRedundancies} method when both USs have benefit part\\
			
			noBenefitInBothUss\newline(testhigHlightRedundancies)&Provision of a CDA report for a US-pair where both USs do not have the \enquote{benefit} part&Search only the main part of USs and mark redundancy clauses with a hash symbol&Verifies the behaviour of the \textit{highlightRedundancies} method when both USs don't have benefit part\\
			
			noBenefitInUs1\newline(testhigHlightRedundancies)&Provision of a CDA report for a US-pair where only the second US have the \enquote{benefit} part&Search only the main part of USs and mark redundancy clauses with a hash symbol&Verifies the behaviour of the \textit{highlightRedundancies} method when only the second US have benefit part\\
			
			noBenefitInUs2\newline(testhigHlightRedundancies)&Provision of a CDA report for a US-pair where only the first US have the \enquote{benefit} part&Search only the main part of USs and mark redundancy clauses with a hash symbol&Verifies the behaviour of the \textit{highlightRedundancies} method when only the first US have benefit part\\
			
			ContainInBenefitPart\newline(testHighlightRedundancies)&Provide a CDA report for a US-pair with redundancy clauses of \enquote{Contains} within \enquote{Benefit} part&The entities included in Contains should be marked with hash symbol&Checks the behaviour of the \textit{highlightRedundancies} method when highlighting redundant entities included in the Contains\\
			
			ContainInMainPart\newline(testHighlightRedundancies)&Provide a CDA report for a US-pair with redundancy clauses of \enquote{Contains} within \enquote{Main} part&The entities included in Contains should be marked with hash symbol&Checks the behaviour of the \textit{highlightRedundancies} method when highlighting redundant entities included in the Contains\\
			
			TargetsInMainPart \newline(testHighlightRedundancies)&Provide a CDA report for a US-pair with redundancy clauses with more than one \enquote{Targets} within the \enquote{Main} part&Founded redundancy clauses should also be marked with a hash symbol, if the main part contains more than one redundancy clause as Targets&Check the behaviour of the \textit{highlightRedundancies} method if there is more than one redundancy clause as Targets in main part\\
			
			TargetsInBenefitPart \newline(testHighlightRedundancies)&Provide a CDA report for a US-pair with redundancy clauses with more than one \enquote{Targets} within the \enquote{Benefit} part&Founded redundancy clauses should also be marked with a hash symbol, if the benefit part contains more than one redundancy clause as Targets&Check the behaviour of the \textit{highlightRedundancies} method if there is more than one redundancy clause as Targets in benefit part\\
			
			
			\hline
				\caption{Test cases for ReportExtractor  class}\label{tb:test_cases_report_extractor}
		\end{tabularx}		
	%	\captionof{table}{Test cases for ReportExtractor  class}\label{tb:test_cases_report_extractor}
		\endgroup
	%\end{figure}

\begin{figure}[h]
	\centering
	\includegraphics[scale=0.3]{code_coverage_report_extractor}
	\caption{Code coverage related to class ReportExtractor}\label{fig:code_coverage_report_extractor}
\end{figure} 

Table \ref{tb:test_cases_evaluation} shows the test cases for the Evaluation.java class and Figure \ref{fig:code_coverage_evaluation} shows the code coverage.

%\newgeometry{margin=2.5cm}
%\begin{landscape}
%\thispagestyle{empty}
%\begin{figure}[h]
\begingroup
\centering
\scriptsize
\renewcommand{\arraystretch}{1,5} 
\keepXColumns

\begin{tabularx}{\textwidth}{X  X  X  X}
	\hline
	Test Case &Supplied Data&Expected Outcome&Description\\
	\hline\hline
	\endfirsthead
	\hline
	Test Case &Supplied Data&Expected Outcome&Description\\
	\hline\hline
	\endhead
	
	testEmptyOrNotExist JsonFile&Provision of a dummy JSON file&Through an exception:\textit{EmptyOrNotExist JsonFile}&Check whether the provided JSON file already exists and is not empty\\
	
	testJsonFile&Provision of a JSON file which is not accessible&Through an exception:\textit{IOException}&Check whether the provided JSON file is accessible\\
	
	main/benefitFull\_Case1&Provide a JSON file with US-pair that have no common textit{Contains} entries but have common \textit{Targets} entries in main/benefit part&In JSON report the entry "Main/Benefit Part Fully Redundant" should be \textit{true}&Check whether US-pair without \textit{Contains} entries can also be fully redundant in main/benefit part\\
	
	main/benefitFull\_Case2&Provide a JSON file with US-pair that have not common textit{Contains} entries but one US has entries as Contains in main/benefit&In JSON report the entry "Main/Benefit Part Fully Redundant" should be \textit{true}&Check whether US-pair without common \textit{Contains} entries can also be fully redundant in main/benefit part\\
	
	main/benefitFull\_Case3&Provide a JSON file with US-pair that have common textit{Contains} entries but one US have additional entries as Contains in main/benefit part&In JSON report the entry "Main/Benefit Part Fully Redundant" should be \textit{true}&Check whether USs with additional \textit{Contains} entries from common contains can also be fully redundant in main/benefit part\\
		
	main/benefitFull\_Case4&Provide a JSON file with US-pair that have exactly common textit{Contains} entries as well as common \textit{Targets} in main/benefit part&In JSON report the entry "Main/Benefit Part Fully Redundant" should be \textit{true}&Check whether USs with \textit{Contains} and \textit{Targets} entries can also be fully redundant in main/benefit part\\
	
	mainBenefitFull&Provide a JSON file with US-pair that all clauses in the main and benefit parts are identical&The JSON report the entries "Main Part Fully Redundant" and "Benefit Part Fully Redundant" should be \textit{true}&Check that all clauses in the main and benefit part are identical, main and benefit part are evaluated as fully redundant\\
	
	mainPartial\_Case1&Provide a JSON file with a US-pair whose entries in \textit{Targets} are partially identical&In the JSON report, the entry "Main Part Partially Redundant" should be \textit{true}&Check whether USs with partially redundant entries in \textit{Targets} are evaluated as partially redundant\\
	
	mainPartial\_Case2&Provide a JSON file with a US-pair whose entries in \textit{Contains} are partially identical&In the JSON report, the entry "Main Part Partially Redundant" should be \textit{true}&Check whether USs with partially redundant entries in \textit{Contains} are evaluated as partially redundant\\
	
	benefitParial&Provide a JSON file with US-pair where only some clauses in \textit{Targets} and \textit{Contains} entries are common&In the JSON report, the entry "Benefit Part Partially Redundant" should be \textit{true}&Check whether USs with partially common \textit{Targets} and \textit{Contains} entries are evaluated as partially redundant\\
	
	mainFullBenefitPartial&Provide a JSON file with US-pair where all clauses in the main part and some in the benefit part are identical&In the JSON report, the entry "Main Part Fully Redundant" and "Benefit Part Partially Redundant" should be \textit{true}&Check whether the main part is evaluated as fully redundant if all clauses are identical and the benefit part is evaluated as partially redundant if some clauses are identical\\
	
	mainPartialBenefitFull&Provide a JSON file with US-pair where some clauses in the main part and all in the benefit part are identical&In the JSON report, the entry "Main Part Partially Redundant" and "Benefit Part Fully Redundant" should be \textit{true}&Check whether the main part is evaluated as partially redundant if some clauses are identical and the benefit part is evaluated as fully redundant if all clauses are identical\\
	
	\hline
	\caption{Test cases for Evaluation  class}\label{tb:test_cases_evaluation}
\end{tabularx}

%\captionof{table}{Test cases for RuleCreator  class}\label{tb:test_cases_rule_creator}
\endgroup
\thispagestyle{empty}
%\end{landscape}
%\restoregeometry

\begin{figure}[h]
	\centering
	\includegraphics[scale=0.45]{code_coverage_evaluation}
	%\includegraphics[scale=0.35]{sequence_diagram}
	\caption{Code coverage related to class Evaluation}\label{fig:code_coverage_evaluation}
\end{figure}
 
\subsection{Evaluation}\label{redundancy_evaluation}
 %	- You have to describe all that in your work, how do you define Main Partial/Full and Benefit Partial/Full, so how do you decide that. And generating this JSON_Report and filling this table is all interesting implementation information.
 
 %When evaluating, I use tools on a large scale and I go through all the cases and look at what comes out of the use case as a meaningful result. As an evaluation, I will enter all the datasets we have as input and then what comes out will be 


\begin{figure}[h]
\center
\includegraphics{Providing_early_feedback_at_the_backlog_level}
\caption{Providing early feedback at the backlog level \cite{mosser2022modelling}}\label{fig:early_feedback}
\end{figure}


 
%\subsection{Evaluation}\label{redundancy_evaluation}
 %	- You have to describe all that in your work, how do you define Main Partial/Full and Benefit Partial/Full, so how do you decide that. And generating this JSON_Report and filling this table is all interesting implementation information.
 
 %When evaluating, I use tools on a large scale and I go through all the cases and look at what comes out of the use case as a meaningful result. As an evaluation, I will enter all the datasets we have as input and then what comes out will be 


\begin{figure}[h]
\center
\includegraphics{Providing_early_feedback_at_the_backlog_level}
\caption{Providing early feedback at the backlog level \cite{mosser2022modelling}}\label{fig:early_feedback}
\end{figure}


 
%\input{Section/Redundancy_Evaluation}


