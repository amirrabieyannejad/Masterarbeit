\section{Introduction}\label{intro}
A \emph{user story} (US) is a brief, semi-structured sentence and informal description of some aspect of a software system that illustrates requirements from the user’s perspective \cite{raharjana2021user}. The brief motivation statement followed the pattern:  As a  \textless\emph{role}\textgreater\ I want to \textless\emph{action}\textgreater\ , so that \textless\emph{value}\textgreater.

\emph{Product backlog} in agile software development acts as a repository for user stories (USs), reflecting the evolving needs and concerns of \emph{system stakeholders}\cite{sedano2019product}. Additionally, the involvement of a \emph{product owner} (PO) in agile development has broken down traditional barriers between development teams and end-users, fostering an environment where the product backlog becomes a central artefact guiding the development process\cite{sedano2019product}. 

However, there is no formal language for expressing stories or modelling backlogs from a practical point of view. Managing redundancies and conflicts between USs is increasingly important, and although agile methodologies such as \emph{scrum} advocate collaboration and daily stand-up meetings to resolve conflicts and redundancies, the process can be resource intensive, especially with extensive backlogs.

Although USs are widely used, there are only a few methods available for evaluating and enhancing their quality.

The IEEE Recommended Practice for Software Requirements Specifications outlines eight key quality attributes for requirements\cite{doe2011recommended}: correct, unambiguous, complete, consistent, ranked for importance/stability, verifiable, modifiable, and traceable. Current methods for evaluating the quality of USs include the heuristics of the INVEST framework\footnote{\href{http://xp123.com/articles/invest-in-good-stories-and-smart-tasks/}{http://xp123.com/articles/invest-in-good-stories-and-smart-tasks/}} (an acronym for independent, negotiable, valuable, estimable, small, and testable) or SMART (an acronym for specific, measurable, assignable, realistic, and time-bound). Lucassen et al. offer a more detailed approach with a Quality User Story (QUS) framework based on a conceptual model for annotating USs\cite{Lucassen2015}, making key information explicit and defining quality dimensions semi-formally. However, this method does not adequately check for redundancies and conflicts.
 %As the development teams navigate through the product backlog to prioritise and sequence the USs, the challenges to efficient implementation and potential conflicts an redundancies will present themselves.

 %The key motivation is to shorten the feedback loop between developers and POs while supporting agile development’s iterative and incremental nature.

In between automated support for extracting domain models from requirements artefacts such as USs play a central role in effectively supporting the detection of redundancies and conflicts between USs. Domain models are a simple way to understand the relationship between artefacts and the whole system. For example, Mosser et al. propose a model engineering method (and the associated tooling) to exploit a graph-based meta-modelling and compositional approach\cite{mosser2022modelling}.

An important gap in current methodologies is the lack of comprehensive redundancy and conflict analysis using graph-based annotations of USs with the main motivation of shortening the feedback loop between developers and POs while supporting the iterative and incremental nature of agile development.
While some tools and techniques exist for managing and prioritizing USs\cite{sachdeva2018prioritizing}\cite{abdelazim2020framework}, they do not adequately address the problem of redundancies between USs in syntactic way. Redundancies can lead to wasted effort, inconsistencies in implementation, and ultimately, a less coherent final product. Effective analysing of redundancies would ensure that similar or overlapping USs are identified and merged or eliminated early in the development process.

Furthermore, there is no framework for systematically analysing conflicts in a semantic way. Instead, the above-mentioned methods such as INVEST or SMART ensure that the USs are well defined. Conflicts between USs can arise for various reasons, e.g. due to overlapping functionalities or conflicting requirements, or ambiguity during defining USs. If left unresolved, these conflicts can significantly hinder the progress of a project. %Although agile practices such as daily stand-ups and sprint reviews promote the resolution of conflicts, they are often not sufficient to recognise and resolve more complex conflicts in a reasonable amount of time.

\emph{Natural language processing} (NLP) techniques offer potential advantages to improve the quality of USs and can be used to parse, extract, or analyse US's conflicts. It has been widely used to help in the software engineering domain \emph{e.g.}, managing software requirements \cite{Arias2018}, extraction of actors and actions in requirement document \cite{al2018use}. 

Furthermore, the incorporation of computational lexicon resources like \emph{VerbNet}\footnote{\href{https://verbs.colorado.edu/verbnet}{https://verbs.colorado.edu/verbnet}} aids in semantic analysis, capturing linguistic and semantic data for a comprehensive understanding.

The overall target of this thesis is to introduce two well-structured workflows, one of which accelerates the automatic detection of potential redundancies using \textit{graph transformation} (GT), \textit{Henshin} and \textit{conflict and dependency analysis} (CDA) tool. Another framework is used to accelerate the automatic detection of potential conflicts between USs expressed in NLP specially VerbNet and an implemented tool.

For this reason, we use the backlogs annotated with Doccano tool\footnote{\href{https://doccano.github.io/doccano}{https://doccano.github.io/doccano}} presented by Mosser et al.\footnote{\href{https://github.com/ace-design/nlp-stories}{https://github.com/ace-design/nlp-stories}} as the primary input \cite{arulmohan2023extracting} and apply the conflicts and redundancies analysis to them.

In order to systematically identify redundancies between USs syntactically, we use the extension CDA from Henshin in addition to model-driven transformation rules. These tools are applied to pairs of USs to determine whether they are \textit{partially} or \textit{fully} redundant based on predefined criteria. This process generates a detailed report highlighting potentially redundant pairs, which is then reviewed by the project team. The team assesses the results and takes the necessary actions accordingly.

For the detection of conflicts between USs in semantic way, we use VerbNet and a specially implemented tool. This approach allows us to recognise conflicts in a comprehensive manner. The generated report lists potential conflict pairs and provides valuable insights to the project team. After receiving the report, the team can review the conflicts and decide on appropriate measures to resolve them.

This paper is organised as follows: The introduction to the US and US quality assurance is given in Section \ref{introduction_to_us}. In Section \ref{preliminaries} we deal with the \textit{extracting domain models from textual requirements}, NLP and \textit{graph transformation tool}. In Section \ref{redundancy}, we comprehensively present our framework related to redundancy analysis. In Section \ref{conflict} we present the second framework namely conflict analysis and conclude with Section \ref{conclusion}.
