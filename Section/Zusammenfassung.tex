%benefit of graph transformation
\section*{Zusammenfassung}
\emph{User Stories}, die Grundbausteine der Softwareentwicklung, dienen als präzise und testbare Beschreibungen der Funktionalität einer Software. Innerhalb des dynamischen Rahmens der \emph{agilen Entwicklung} werden diese User Stories in der Regel informell in einfachem Text verfasst und im \emph{Product Backlog} verwaltet, das als Repository für die Priorisierung und Verfolgung von Entwicklungsaufgaben dient.

\emph{Behaviour Driven Development} (BDD), ein spezieller Ansatz im Bereich der agilen Softwareentwicklung, legt einen starken Schwerpunkt auf die iterative Umsetzung von User Stories. Die Reihenfolge der User Stories im BDD ist ein zentraler Aspekt der Methodik. Die richtige Reihenfolge hat nicht nur Auswirkungen auf die Effizienz der Entwicklung, sondern auch auf den Gesamterfolg des Projekts. Durch eine effektive Priorisierung und Abfolge der User Stories können die Entwicklungsteams den Benutzern einen inkrementellen Mehrwert bieten, auf sich ändernde Anforderungen reagieren und sicherstellen, dass die kritischsten Funktionen zuerst behandelt werden.

User Stories sind oft voneinander abhängig, was zu potenziellen Konflikten führen könnte, wenn eine User Story die Löschung einer Komponente erfordert, die für die erfolgreiche Ausführung einer anderen User Story unerlässlich ist, oder wenn eine User Story ein Element einführt, das der Realisierung einer anderen User Story zuwiderläuft und diese somit verhindert. Darüber hinaus kann auch die Änderung einer bestehenden Anforderung oder das Hinzufügen einer neuen Anforderung zum bestehenden Product Backlog in der agilen Softwareentwicklung zu Konflikten führen, da sich die Bedürfnisse und Anliegen der Systemstakeholder ändern. Dies kann zu einer Vielzahl von Inkonsistenzen aufgrund widersprüchlicher Anforderungen führen, da Anforderungen von mehreren an der Produktentwicklung beteiligten Stakeholdern gestellt werden, um unterschiedliche Funktionen zu erreichen.

Um das Auftreten von Konflikten zu minimieren, sollten Teams in der Regel systematisch Abhängigkeiten zwischen User Stories innerhalb ihres Backlogs identifizieren und dokumentieren. Agile Methoden wie \emph{Scrum} fördern die funktionsübergreifende Zusammenarbeit und tägliche Stand-up-Meetings als Mechanismen, um Abhängigkeiten und Konflikte zeitnah anzugehen und zu entschärfen. Dieser Ansatz kann jedoch zeit- und ressourcenaufwendig sein. In Fällen, in denen das Backlog sehr umfangreich ist, kann das Erkennen bestehender Konflikte zwischen User Stories zu einem komplexen Unterfangen werden.

\emph{Modellbasiertes Software-Engineering} ist eine geeignete Methode, um die ständig steigende Komplexität von Softwareentwicklungsprozessen zu bewältigen. \emph{Graphen} und \emph{Graphtransformationen} haben sich als nützlich erwiesen, um solche Modelle und deren Änderungen zu visualisieren.

Da es in der agilen Softwareentwicklung keinen Prozess zur systematischen Identifizierung und Verwaltung von Konflikten zwischen in natürlicher Sprache erstellten Anforderungen gibt, soll in dieser Arbeit ein gut strukturierter Arbeitsablauf vorgestellt werden, der eine Sammlung von Techniken und Werkzeugen aus verschiedenen Bereichen verwendet, um die automatische Erkennung von Konflikten und Abhängigkeiten in natürlichsprachlichen User Stories zu beschleunigen.

