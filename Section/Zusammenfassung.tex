%benefit of graph transformation
\section*{Zusammenfassung}
\emph{User Stories} (USs), die Grundbausteine der Softwareentwicklung, dienen als präzise und testbare Beschreibungen der Funktionalität einer Software. Im dynamischen Rahmen der \emph{agilen Entwicklung} werden diese USs zu einer häufig verwendeten Anforderungsnotation in agilen Projekten\cite{wang2014role}, die in der Regel informell in Klartext geschrieben und im \textit{product backlog} verwaltet wird, das als Repository für die Priorisierung und Verfolgung von Entwicklungsaufgaben dient.

Wenn die Anzahl der USs zunimmt, sind Konflikte oder Redundanzen zwischen ihnen unvermeidlich. Wenn eine User Story (US) die Streichung einer Komponente erfordert, die für die erfolgreiche Ausführung einer anderen US unerlässlich ist, haben wir es mit einem Konflikt zu tun, oder wenn eine US (oder einige Elemente oder Teile davon) eine syntaktische Duplikation einer anderen US ist, haben wir es mit Redundanz zu tun.

Darüber hinaus kann die Änderung einer bestehenden Anforderung oder das Hinzufügen einer neuen Anforderung zum bestehenden Product Backlog in der agilen Softwareentwicklung aufgrund von Änderungen der Bedürfnisse und Anliegen der Systembeteiligten ebenfalls zu Konflikten oder Redundanzen führen. Dies kann zu einer Vielzahl von Inkonsistenzen führen, da Anforderungen von mehreren an der Produktentwicklung beteiligten Interessengruppen gestellt werden, um unterschiedliche Funktionen zu erreichen.

Die effektive Erkennung dieser Konflikte und Redundanzen ist für Entwicklungsteams von grundlegender Bedeutung. Indem sie sich mit diesen Problemen auseinandersetzen, können die Teams den Benutzern einen zusätzlichen Nutzen bereitstellen, sich an veränderte Anforderungen anpassen und die Konsistenz zwischen den USs aufrechterhalten. 

Normalerweise fördern agile Methoden wie \textit{Scrum} die funktionsübergreifende Zusammenarbeit und tägliche Stand-up-Meetings als Mechanismen, um Redundanzen und Konflikte zeitnah anzugehen und zu entschärfen. Dieser Ansatz kann jedoch zeit- und ressourcenaufwendig sein. In Fällen, in denen der Rückstand sehr groß ist, kann das Erkennen von Redundanzen und Konflikten zwischen USs zu einem komplexen Unterfangen werden.

Da es in der agilen Softwareentwicklung keine Methode zur automatischen Erkennung von Redundanzen und Konflikten zwischen in natürlicher Sprache verfassten Anforderungen gibt, möchten wir in dieser Arbeit zwei Ansätze zur Analyse von Redundanzen und Konflikten zwischen USs vorstellen.

Der erste Ansatz analysiert Redundanzen, indem er annotierte USs als Input nimmt (anstelle von US-Text) und \textit{graph transformation} (GT), insbesondere \textit{Henshin} und sein \textit{Conflict and Dependency Analysis} (CDA) Tool, verwendet, um potenzielle Redundanzen auf syntaktische Weise zu erkennen.

Der zweite Ansatz befasst sich mit der Analyse von Konflikten zwischen USs und verwendet ebenfalls annotierte USs als Input. Durch die Verwendung von \textit{Natural Language Processing} (NLP)-Techniken, insbesondere VerbNet, und einem speziell implementierten Tool, erkennen wir potenzielle Konflikte auf semantische Weise.

Wir wenden beide Ansätze auf 19 annotierte Backlog-Datensätze an und erstellen umfassende Berichte für jeden Datensatz. Bei der Auswertung stellen wir fest, dass die Ergebnisse beider Ansätze zufriedenstellend sind. Wir stellen jedoch auch fest, dass die Qualität der USs und ihrer Annotationen die Effektivität der Ergebnisse erheblich beeinflusst.